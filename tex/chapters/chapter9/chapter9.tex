%=================================================================
\chapter{TRABAJO RELACIONADO} \label{cha:relatedwork}
%=================================================================

Cuenca et al. \cite{Cuenca:2010} proponen un completo marco de modelamiento para la alineaci\'on Negocio-IT, que incluye las fases del ciclo de vida, modelo de madurez, vistas y artefactos. Su implementaci\'on se realiza a trav\'es de entrevistas y encuestas a expertos. En \cite{Plazaola:2008} se direcciona la evaluaci\'on de alineaci\'on Negocio-IT sobre EA como entrada para mejorar la toma de decisiones de alineamiento. Un metamodelo permite expresar criterios de calidad y evaluarlos utilizando reglas de inferencia para cuantificar un nivel de madurez de la organizaci\'on. La recolecci\'on de los datos del modelo de entrada se hace a mediante encuestas. Elhari y Bounabat \cite{Elhari:2011} presentan una plataforma para evaluar la alineaci\'on usando EA basada en un conjunto de m\'etricas que determinan un nivel de madurez de alineaci\'on. Este trabajo se soporta en la aplicaci\'on de encuestas para calcular dichas m\'etricas. Los datos de entrada de los modelos anteriores se resumen en encuestas y entrevistas a expertos. Nuestra propuesta, en cambio, toma como datos de entrada las definiciones formales de la EA para inferir trazas y medir la alineaci\'on, de tal manera que esperamos obtener resultados m\'as puntuales y objetivos. Kalcas actualmente tiene un alcance mas reducido dado que se focaliza solo en BA e IA y no en todos los dominios de la EA, pero Kalcas aporta mayor automatizaci\'on al inferir trazabilidad. 

En \cite{Aversano:2005} se propone a estrategia para ser aplicada durante la evoluci\'on para detectar desalineamientos entre procesos de negocio y los IS que los soportan. La estrategia es identificar los objetos que deben ser modificados para restaurar el alineamiento y considera un conjunto de atributos que representan indicadores de posible desalineamiento basados en an\'alisis de impacto. Una grafo de dependencia entre procesos de negocio e IS debe existir previamente para posibilitar el an\'alisis de impacto. Este grafo de dependencia es an\'alogo a la trazabilidad que buscamos inferir con nuestra aproximaci\'on. Kalcas no propone ajustes para restaurar el alineamiento solo evidencia los posibles desalineamientos entre BA-IA de una manera automatizada. En nuestro caso, el grafo de dependencia no est\'a dado, sino es inferido a trav\'es del matching de ontolog\'ias.

En \cite{Wegmann:2005} se introduce el marco te\'orico SEAM (Systemic Enterprise Architecture) y una herramienta asociada (SeamCAD) que chequea la alineaci\'on a trav\'es de la comparaci\'on de modelos descritos en t\'erminos de jerarqu\'ias funcionales y organizacionales. Esta propuesta requiere la definici\'on expl\'icita de las relaciones entre los diferentes componentes. La terminolog\'ia de SEAM est\'a basada en \textit{Reference Model for Open Distributed Processing} (RM-ODP) e implementada en una ontolog\'ia que permite usar m\'etodos formales de razonamiento. SEAM es aplicado en tiempo de dise\~no de la arquitectura para promover el alineamiento. La propuesta de Wegmann al igual que Kalcas utiliza razonamientos basados en ontolog\'ias de dominio espec\'ifico. Pero Kalcas est\'a enfocado en an\'alisis sobre BA e IA ya construidas, no pretende ser un nuevo lenguaje con el cual modelar arquitecturas de datos y procesos. Adicionalmente, Kalcas puede ser aplicado tanto en tiempo de dise\~no como de ejecuci\'on, dado que puede utilizarse sobre la arquitectura actual u objetivo.

Un framework propuesto por el grupo CEO (\textit{Centro de Engenharia Organizacional}) se describe en \cite{Vasconcelos:2007} como un conjunto de primitivas de modelamiento para expresar EA, Arquitectura de IS, Arquitectura de Software y las dependencias entre entre ellas. Adem\'as se definen un conjunto de m\'etricas que se eval\'uan autom\'aticamente para darle al arquitecto un conjunto de indicadores sobre el impacto de cada una de sus decisiones durante el proceso de construir una Arquitectura de IS. El Framework CEO apunta a proveer una descripci\'on formal de objetivos de negocio, procesos, recursos y SI. El lenguaje de modelamiento usado para implementar CEO fue UML, por lo cual se extendi\'o a trav\'es de estereotipos para hacer mas clara la identificaci\'on de los conceptos de negocio. Kalcas est\'a soportado en un metamodelo de EA, por tanto tambi\'en hace uso de un lenguaje para la expresi\'on de conceptos del dominio de la EA. Desde ese punto de vista, CEO tiene la ventaja de ser compatible con UML requerir\'ia transformaciones adicionales para hacerlo compatible con UML y poder ser utilizado por herramientas que soportan dicho lenguaje. A pesar que Kalcas tiene la informaci\'on necesaria para calcular m\'etricas, de momento no genera indicadores de alineaci\'on. De otro lado, Kalcas mas que un modelador, utiliza ingenier\'ia inversa sobre modelos ya existentes e infiere las relaciones entre tales modelos.

El trabajo de \cite{Simonin:2007} est\'a enmarcado en el desarrollo iterativo de servicios en \textit{France-Telecom}, el cual est\'a basado en \textit{Unified Process} (UP). La propuesta se apoya en MDA para medir en cada iteraci\'on de dise\~no de servicio el alineamiento entre requerimientos funcionales y la IA. Cuando se dise\~nan los servicios, los requerimientos funcionales son asociados a las entidades de negocio por el arquitecto. M\'etricas de alineamiento son propuestas para evaluar la perdida de informaci\'on entre el an\'alisis de requerimientos funcionales y el dise\~no de la IA. Estas m\'etricas son usadas en el contexto de un proceso de dise\~no iterativo para ayudar al arquitecto a seleccionar la mejor soluci\'on entre dos iteraciones. En com\'un, ambas propuestas involucra el modelo de la IA, pero Kalcas no aborda el an\'alisis de requerimientos sino los procesos de negocio de la compa\~nia. Nuestro trabajo asume que el dise\~no de la EA no incluy\'o un mecanismo de trazabilidad inicialmente, sino pretende inferir la trazabilidad en una EA ya construida. El uso de MDA para formalizar las definiciones est\'a presente tanto en \cite{Simonin:2007} como en Kalcas, pero el primero se apoya en reglas de transfomaci\'on de modelos para hallar inconsistencias, mientras que nuestro trabajo utiliza el razonamiento basado en ontolog\'ias. 

ArchiMate \cite{Jonkers:2004} es uno de los lenguajes de modelamiento de integraci\'on de EA mas difundidos y permite expresar a trav\'es de notaciones las relaciones que se dan entre los componentes de diferentes dominios de la EA. Nuestra aproximaci\'on no asume como preexistente la trazabilidad entre los dominios arquitecturales, sino pretende inferirlas a partir de las definiciones contenidas en la EA. Archimate aborda trazabilidad entre todos los dominios de la EA, Kalcas en su versi\'on actual solo incluye elementos de BA e IA. Y como en los casos anteriores, nuestro objetivo es ofrecer una herramienta para la evaluaci\'on del un modelo existente y no un lenguaje de modelamiento como tal.

En \cite{Pereira:2003, Pereira:2005, Sousa:2005} se eval\'ua la alineaci\'on mediante la verificaci\'on de heur\'isticas propuestas que deben cumplirse entre los diferentes dominios de la EA. Por ejemplo entre procesos y dato: Todos los procesos crean, actualizan y/o borran al menos una entidad; Todas las entidades son le\'idas al menos por un proceso; Entre procesos y aplicaciones: Cada proceso de negocio deber\'ia ser soportado por al menos una aplicaci\'on; Las tareas de los procesos de negocio deber\'ian ser soportadas por una sola aplicaci\'on; Procesos de negocio cr\'iticos deber\'ian depender de aplicaciones escalables y con alta disponibilidad. Aunque el trabajo propone heur\'isticas entre los dominios de IA, BA, AA y TA, no aborda una herramienta para realizar verificaciones de estas heur\'isticas, luego asumen revisiones manuales. Nuestra definici\'on de desalineaci\'on se basa en las heur\'isticas presentadas en estos trabajos, limit\'andose a evaluar solo a BA e IA. Nuestro trabajo habilita la expresi\'on y evaluaci\'on de estas heur\'isticas mediante KQL apoyando as\'i las tareas del arquitecto.

Otros trabajos han propuesto diferentes mecanismos para comparar componentes de la BA. En \cite{Remco:2009} se propone la aplicaci\'on de alineaciones basadas en grafos y l\'exicos para encontrar actividades similares en modelos de procesos de negocio. Por otro lado, una propuesta para expresar procesos de negocios con redes de Petri sobre ontolog\'ias se expone en \cite{Brockmans:2006} y su objetivo es realizar alineaciones sem\'anticas para soportar interconectividad semiautom\'atica de procesos de negocio. Por otro lado en \cite{Rodriguez:2011} se hallan diferencias entre dos versiones de un mismo proceso de negocio calculando el delta entre modelos. La coincidencia de estos trabajos con Kalcas, radica en que todos permiten detectar similitudes entre procesos de negocio utilizando modelos enriquecidos. Adicionalmente, con Kalcas exploramos el uso de matching de ontolog\'ias para hallar coincidencias entre activos de informaci\'on y relaciones de trazabilidad entre entre BA e IA. Junto con el trabajo de Rodr\'iguez \cite{Rodriguez:2011} compartimos el metamodelo Tartarus como n\'ucleo de nuestra propuesta, pero nuestros razonamientos utilizan ontolog\'ias a cambio de encontrar diferencias entre modelos.

La investigaci\'on documentada en \cite{Murcia:2011} tambien se apoy\'o en el metamodelo Tartarus para inferir redundancias dentro de una IA. Al igual que nuestra propuesta, se soport\'o en matching de ontolog\'ias para hallar tales redundancias, en particular el motor de alineamiento utilizado fue AgreementMaker \cite{Cruz:2009}. El trabajo all\'i presentado fue una fase previa a Kalcas y tienen en com\'un el uso de ontolog\'ias para inferir relaciones. Una vez hechas las inferencias, estas se presentan a manera de tablas HTML indicando el porcentaje de similitud. Dado que Kalcas es una continuaci\'on del trabajo propuesto por Murcia, nosotros reutilizamos algunos elementos como el importador de IA y la transformaci\'on de IA a ontolog\'ias. Comparado con Kalcas, el trabajo se limita a la detecci\'on de redundancias en IA, no aborda el concepto de alineamiento y carece de una herramienta para dise\~nar y ejecutar consultas sobre los modelos Tartarus.

El trabajo de tesis publicado en \cite{Moya:2012} aborda el alineamiento de Negocio e IT enfocado en la definici\'on de relaciones entre los pilares de Ross \cite{Ross:2006} y los conceptos presentados por Henderson y Venkatraman \cite{henderson:1990}. Una aproximaci\'on MDA de los pilares: Modelo Operacional, EA e Infraestructura de IT permite la definici\'on de alineamiento como el conjunto de relaciones que existe entre tales pilares. Al igual que Kalcas, el trabajo incluye la definici\'on de un DSL gr\'afico que soporta el dise\~no y ejecuci\'on de consultas de alineamiento sobre el metamodelo \textit{Tarmivol}. \textit{Tarmivol} permite integrar los metamodelos \textit{Tartarus}, \textit{Millo} (Metamodelo de procesos de negocio) y \textit{Archivol} (Metamodelo de arquitecturas de soluci\'on). La propuesta de Tarmivol tiene un espectro mas amplio comparada con Kalcas, dado que incluye los dem\'as dominios de la EA. Pero por otro lado es menos efectiva al inferir las relaciones de trazabilidad y alineaci\'on  porque solo tiene en cuenta elementos que sean textualmente exactos. Kalcas en cambio, se apoya en razonadores que logran inferencias mas complejas entendiendo que en el mundo real los componentes de BA e IA dif\'icilmente poseen nombres exactamente iguales.
