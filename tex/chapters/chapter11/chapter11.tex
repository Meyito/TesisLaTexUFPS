%=================================================================
\chapter{SINZA} \label{cha:conclusions}
%=================================================================

En este cap\'itulo discutimos SOBRE SINZA%**************************************************************
\section{Trabajo Futuro} \label{sec:futurework}
%**************************************************************

Las siguientes son algunas propuestas de trabajo futuro que hemos visualizado como continuaci\'on o extensiones a nuestra propuesta:

\begin{itemize}

\item Futuras investigaciones pueden incorporar la definici\'on de m\'etricas de alineaci\'on que permitan evaluar una EA frente a niveles de madurez presentados en trabajos anteriores como \cite{Elhari:2011}.

\item La inclusi\'on de los dem\'as dominios de la EA como Aplicaciones (AA) y Tecnolog\'ia (TA) mejorar\'ia la completitud de esta propuesta. Inferir trazabilidad con elementos que hacen parte de la estrategia de la organizaci\'on como drivers, principios, objetivos aportar\'ia acercar\'ia mas nuestra propuesta a la alineaci\'on negocio-IT. En este punto hay consideraciones importantes a investigar, como el hecho de alinear elementos a diferente nivel de granularidad, como lo son por ejemplo un driver de negocio y una entidad de informaci\'on.

\item Los mapeos generados con este framework podr\'ian exportarse a lenguajes est\'andar de integraci\'on de EA como ArchiMate \cite{Jonkers:2004}. De tal manera que diferentes herramientas puedan reutilizar las inferencias alcanzadas con Kalcas. Otros tipos de consultas que resuelvan diferentes preguntas sobre el modelo pueden enriquecer el editor KQL, por ejemplo obtener todas las actividades o entidades que no est\'en alineadas.

\item Las tareas de alineamiento en la fase de experimentaci\'on han utilizado algunos algoritmos de Alignment API, extender las pruebas en cuanto a motores y algoritmos para evaluar mas resultados puede mejorar la exactitud promedio de Kalcas. Para el momento de nuestro trabajo no hay disponible un algoritmo ling\"u\'istico aplicable al idioma espa\~nol, por tanto un avance en esa direcci\'on es muy valioso no solo para esta propuesta sino para el campo del matching de ontolog\'ias en general. 

\item En este trabajo se utiliz\'o la trazabilidad inferida para apoyar an\'alisis de alineaci\'on, pero nuevas investigaciones pueden emplear la trazabilidad para soportar el an\'alisis de impacto en una EA. Nuestra propuesta podr\'ia ser extendida o modificada para ser usada con otros metamodelos de EA diferentes a Tartarus.

\end{itemize}

%**************************************************************
\section{Publicaciones} \label{sec:publish}
%**************************************************************
Durante toda la realizaci\'on de este trabajo de tesis, fueron aceptadas publicaciones en modalidad \textit{full research papers} en los siguientes eventos internacionales:

\begin{itemize}

\item 16th East-European Conference on Advances in Databases and Information Systems
(ADBIS 2012): KALCAS: A frameworK for semi-Automatic aLignment of data and business proCesses ArchitectureS. Poznan, Polonia. Septiembre 2012.

\item XXX International Conference of the Chilean Computer Science Society (SCCC 2011): An Ontology-Matching based Proposal to Detect Potential Redundancies on Enterprise Architectures. Curic\'o, Chile. Noviembre 2011.

\item XXXVII Conferencia Latinoamericana de Inform\'atica (XXXVII CLEI): Detecci\'on de Elementos Redundantes en
Arquitecturas de Informaci\'on: Un Enfoque Apoyado en Alineaci\'on de Ontolog\'ias. Quito, Ecuador. Octubre 2011.

\end{itemize}


