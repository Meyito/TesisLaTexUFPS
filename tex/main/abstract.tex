%------------------------------------------------------------------
%         Abstract
%------------------------------------------------------------------
\begin{abstract}

The alignment of Business Processes and Information Technologies (IT) is among the top concerns of IT management surveys, because it has a direct impact on the organization's agility and flexibility to change in response to business needs. Enterprise Architecture (EA) is a valuable instrument to assess and achieve such alignment. EA frameworks describe the organization in domains like Business (BA), Information (IA), Application (AA) and Technology Architecture (TA). Different works have proposed frameworks and methodologies for alignment evaluation across elements contained in EA domains; however, they suppose manual tasks such as applying surveys or comparing EA's artifacts. These manual tasks applied over heterogeneous artifacts are time-costly, error-prone and impractical on large EAs. We introduce KALCAS, a framework to support the alignment between BA and IA via the automatic comparison of their constituent elements supported in Model Driven Architecture (MDA) and Ontology Matching techniques. Our key objectives are: i) To support the process of evaluating BA-IA alignment; ii) to automatically discover traceability among the elements in the BA and IA domains; iii) to detect potential alignments and misalignments between BA and IA. We present the obtained results of this approach applied in the Institute responsible for assessing the quality of education in Colombia.

\textit{Keywords}: Business-Technology Alignment, Semi-Automatic Traceability Detection; Alignment Heuristic, Ontology Matching, Enterprise Architecture Model.


\end{abstract}