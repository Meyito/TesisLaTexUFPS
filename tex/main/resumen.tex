\phantomsection
\addcontentsline{toc}{chapter}{Resumen}
\chapter*{Resumen\markboth{Resumen}{}}

El alineamiento entre procesos de negocio y tecnolog\'ias de informaci\'on (IT) es la preocupaci\'on principal en los estudios de gerencia de IT, debido al impacto directo que ejerce sobre la agilidad y flexibilidad de las organizaciones. La arquitectura empresarial (EA) es un valioso instrumento para evaluar y alcanzar tal alineamiento. Los marcos de EA describen la organizaci\'on en dominios como Arquitectura de Negocio, Informaci\'on, Aplicaciones y Tecnolog\'ia. Diferentes trabajos han propuesto marcos y metodolog\'ias para evaluar la alineaci\'on a trav\'es de los elementos contenidos en los dominios de EA. Sin embargo, suponen tareas manuales como aplicar encuestas o comparar artefactos de EA. Estas tareas manuales aplicadas sobre artefactos heterog\'eneos son costosas en tiempo y recursos, propensa a errores e impr\'acticas en extensas EAs. Presentamos KALCAS, una propuesta para soportar alineamiento entre BA e IA a trav\'es de comparaci\'on autom\'atica de sus elementos constitutivos y soportado en Arquitectura Dirigida por Modelos (MDA) y Matching de Ontolog\'ias. Nuestros principales objetivos son: i) Soportar el proceso de evaluaci\'on de alineamiento entre BA e IA y ii) Automatizar la detecci\'on de potenciales alineamientos y desalineamientos  entre BA e IA. Presentamos los resultados obtenidos en la aplicaci\'on de esta aproximaci\'on en el Instituto Colombiano para la Evaluaci\'on de la Educaci\'on (ICFES).

\textit{Keywords}: Alineamiento de Negocio y Tecnolog\'ia, Detecci\'on Semi-Automatica de Trazabilidad; Heur\'isticas de Alineaci\'on, Matching de Ontolog\'ias, Modelo de EA.

